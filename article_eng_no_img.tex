\documentclass[aps,preprint]{revtex4}%
\usepackage{amsfonts}
\usepackage{amsmath}
\usepackage{amssymb}
\usepackage{graphicx}
\usepackage[T2A]{fontenc}
\usepackage[cp1251]{inputenc}
\usepackage{srcltx}%
\setcounter{MaxMatrixCols}{30}
%TCIDATA{OutputFilter=latex2.dll}
%TCIDATA{Version=5.00.0.2606}
%TCIDATA{CSTFile=revtex4.cst}
%TCIDATA{Created=Thursday, April 14, 2005 22:08:00}
%TCIDATA{LastRevised=Monday, June 22, 2015 08:59:31}
%TCIDATA{<META NAME="GraphicsSave" CONTENT="32">}
%TCIDATA{<META NAME="SaveForMode" CONTENT="1">}
%TCIDATA{BibliographyScheme=Manual}
%TCIDATA{<META NAME="DocumentShell" CONTENT="Articles\SW\REVTeX 4">}
\makeatletter
\renewcommand{\thesection}{\arabic{section}}
\renewcommand{\p@subsection}{}
\renewcommand{\thesubsection}{\arabic{section}.\arabic{subsection}}
\renewcommand{\p@subsubsection}{}
\renewcommand{\thesubsubsection}
{\arabic{section}.\arabic{subsection}.\arabic{subsubsection}}
\newcommand{\dumfig}[1]
{\begin{figure}\caption{#1}\label{#1}\end{figure}}
\makeatother
\begin{document}
Model of complex networks with rejection

\bigskip

Igor Zarvanskii, Andrei Snarskii

\bigskip

A modification of the preferential attachment rule, the attachment with
rejection, is proposed and applied to network models constructed according to
the Barab\'{a}si-Albert algorithm and to $(u,v)$-flowers. The results of
numerical simulation for the considered models are reported, and various
parameters of simulated networks are analyzed. The parameters of obtained
networks are shown to behave similarly to the phase transitions of the second
kind. The threshold value of the rejection parameter, at which the phase
transition takes place, has been calculated.

\bigskip

\section{Introduction}

A considerable number of real complex networks are scale-free, i.e. the
degrees of network nodes are distributed according to the power law. Such
networks include WWW networks, metabolic network, food webs, social networks,
and many others \cite{Dor2}.

Nowadays, the properties of such scale-free networks have been studied in
detail, and the network parameters (the average node degree, the shortest
average path length, the clustering coefficient, and so forth) have been
determined \cite{Newman1}. It should be noticed that complex networks
constructed according to work \cite{AlBa2} are an idealization of real
networks, and the parameters of the latter sometimes can considerably differ
from perfect ones \cite{Newman1}. Nevertheless, the power-law dependence for
node degrees in real complex networks is rather widespread, especially for the
networks that are formed by processes evolving in time \cite{Newman3, AlBa3}.

One of such processes is the redistribution of \textquotedblleft
wealth\textquotedblright\ (as such, one may take money, investments, real
estate, and so on) among people. Its study began long before the concept of
complex network appeared. V.~Pareto formulated the power-law distribution of
wealth (the so-called Pareto law) \cite{Pareto}: the fraction $\nu$ of
population is the power-law function of the fraction $\mu$ of wealth owned by
them, $\nu\sim\mu^{\gamma}$. At $\gamma=0.86$, we obtain that 20\% of
population own 80\% of wealth, so that the Pareto law is often called the 80/20.rule.

In work \cite{AlBa1}, an algorithm for the formation of a complex network with
the power law of node degree distribution (the so-called Barab\'{a}si-Albert
algorithm) was found. It is based on two essentially important points:

\noindent(i)~the network is growing; starting from a definite node number
$m_{0}$, at every timestep there appear a certain number of new nodes with $n$ edges;

\noindent(ii)~the probability for the edges of a new node to be attached to
the existing ones is proportional to the node degree.

\noindent Briefly speaking, the Barab\'{a}si-Albert model is a growing network
with the preferential attachment.

Later, a lot of modifications of the Barab\'{a}si-Albert algorithm were
proposed. About 20 of them are indicated in work \cite{AlBa2}. All those
modifications give rise to scale-free networks with various indices of node
distribution over the node degree. At first sight, the growing network with
various types of preferential attachment inevitably has to transform into a
scale-free network.

In this work, we demonstrate that a slight and, at first sight, insignificant
modification of the law of preferential attachment is possible, at which the
power-law distribution given by the Barab\'{a}si-Albert model becomes
violated. In this case, the distribution function has a discontinuity, which
means that there are no network nodes which degrees fall within a certain
interval of values. Detailed researches showed that the introduced parameter
$r$, which describes the modification of the preferential attachment law, has
a threshold value $r_{c}$, so that the network remains scale-free at $r<r_{c}%
$, whereas a discontinuity in the node degree distribution function emerges at
$r\geq r_{c}$. The magnitude of discontinuity is a power-law function of the
distance between the parameter $r$ and its threshold value, which gives a
ground to talk about an analogy with the order parameter behavior at the phase
transition of the second kind.

We also verified the modification of the attachment law using deterministic
hierarchical scale-free networks, the so-called $(u,v)$-flowers \cite{Dor1}.
In this case, the violation of the power-law distribution function was
observed, the behavior of which was similar to that of the order parameter at
the phase transitions of the second kind.

\section{Barab\'{a}si-Albert model with rejection}

\subsection{Barab\'{a}si-Albert algorithm}

Consider a growing network. In the standard variant of Barab\'{a}si-Albert
model \cite{AlBa1}, there exist $m_{0}$ nodes at the first timestep, which are
connected with each other. At every next step, there arises $m$ new nodes,
each with $q$ edges. The quantity
\begin{equation}
p_{i}=\frac{k_{i}}{\sum\limits_{j}k_{j}}\label{Pi}%
\end{equation}
characterizes the probability of attachment (the creation of an edge between
the nodes) of a new node to the already existing $i$-th node; it is
proportional to the $i$-th node degree $k_{i}$ (the number of its edges), and
summation is carried out over all \textquotedblleft old\textquotedblright\ nodes.

When the number of timesteps is large, this algorithm gives rise to a
power-law distribution function of node degrees $P(k)$:%
\begin{equation}
P(k)\sim k^{-\gamma}, \label{eq:powerlaw}%
\end{equation}
with the exponent $\gamma=3$ \cite{AlBa1}. In work \cite{AlBa2}, a good many
other modifications of the preferential attachment rule are quoted, which
bring about various values of exponent $\gamma$. However, the power-law
dependence \eqref{eq:powerlaw} survives.

\subsection{Modification of Barab\'{a}si-Albert algorithm}

In this work, we propose to generalize the model based on the preferential
attachment rule introduced by Barab\'{a}si and Albert. The new criterion will
be referred to as the rule of preferential attachment with rejection. In the
framework of this model, a new preferential attachment rejection parameter
$r$, which acquires values in the interval $[0,1]$, is introduced. In
particular, when node $i$ with the degree $k_{i}$ is chosen to form an edge
with the new node, a new attachment occurs with probability (\ref{Pi}), i.e.
in the case if the condition
\begin{equation}
k_{i}\geq r\langle k\rangle\label{eq:exceptive}%
\end{equation}
is satisfied, where $\langle k\rangle=\frac{1}{N}\sum\nolimits_{j}{k_{j}}$\ is
the average value of node degrees in the network at the moment immediately
before the probable attachment. In other words, only attachments to
\textquotedblleft rich\textquotedblright\ nodes, which degrees are not less
than $r\langle k\rangle$, are allowed. Extra condition \eqref{eq:exceptive}
deactivates some nodes in the course of network growth, i.e. they cannot form
edges with the new node at the given timestep. It should be noticed that, if a
certain node does not satisfy condition \eqref{eq:exceptive} at the given
moment, it does not mean that new nodes cannot be attached to it later,
because the $\langle k\rangle$-value changes in time.

\subsection{Distribution function for node degrees}

If the rejection parameter $r=0$, the proposed model transforms into the
standard Barab\'{a}si-Albert model, because $k_{i}$ is always larger than
zero. Unexpectedly, the rejection parameter was found to have a threshold
behavior. If the rejection parameter is less than a certain threshold value
$r_{c}$, i.e. at $r<r_{c}$, the distribution function of node degrees $P(k)$
remains to be power-law and, therefore, the network itself remains scale-free.
At the rejection parameter values larger than the threshold one ($r\geq r_{c}%
$), the network changes its structure; namely, nodes with \textquotedblleft
intermediate\textquotedblright\ node numbers disappear, which can be seen in
Fig.~\ref{fig:rankDistribution-rank}.

Let us determine the threshold value of rejection parameter. For this purpose,
we calculated a network with the initial node number $m_{0}=20$. At every
step, there emerged one node with $q=3$ edges. After 80~steps, we obtained a
network with $N=100$ nodes. Repeating this procedure many times for various
$r$'s , we found an $r$-value at which the distribution of $p_{i}$ for the
specific network ceased to be power-law, i.e. when there appeared a break in
the network. The results of calculations demonstrated that $r_{c}(100)=0.62$
at $N=100$. As $N$ increased from 100 to 2000, the value of $r_{c}$ decreased
and saturated at $r_{c}=0.51\pm0.04$. This value of $r_{c}$ was considered to
be the threshold value of rejection parameter for large (infinite) networks.
The threshold value of rejection parameter did not change if the initial
parameters $m_{0}$ and $q$ are varied. In further calculations, the indicated
$r_{c}$-value was used.

Let us introduce a new characteristic of the network, the gap magnitude $\eta$
(see Fig.~\ref{fig:rankDistribution-rank}), i.e. the smallest magnitude of the
difference between the node degree values to the left and to the right of the
discontinuity in the figure. The gap interval marks the node degree values
(the ordinate axis) that are absent in the network.

The behavior of the parameter $\eta$ is similar to that of the order parameter
in the theory of phase transitions of the second kind \cite{Landau}. It is
known that, when the temperature approaches the critical value $T_{c}$, the
order parameter $\eta$---e.g., it may be the magnetization---diminishes
following the power law $\eta\sim(T-T_{c})^{\beta}$, where $\beta$ is the
critical index.

For the numerical experiment, the following parameters were chosen: the final
node number $N=5000$, the initial node number $m_{0}=20$, and the edge number
for every new node $m=3$. In Fig.~\ref{fig:rankDistribution-gap}, the obtained
dependence $\eta=A\,{(r-r_{c})}^{\beta}$, where $\beta=1.15\pm0.05$, is shown.

\subsection{Clustering and assortativity coefficients}

The appearance of gap $\eta$ in the node degree distribution function $P(k)$
testifies to a substantial variation in the network structure, which has to
manifest itself in the network parameters. Let us consider the behavior of the
clustering, $C$, and assortativity, $A$, coefficients as functions of the
rejection coefficient $r$ in the vicinity of $r=r_{c}$. For a network with
$N=5000$ nodes, the both do not depend on $r$ at $r<r_{c}$; they equal
$C_{0}\approx0.01$ and $A_{0}\approx-0.096$, and coincide with the values
calculated in works \cite{AlBa2,Newman2}. For $r$ growing from $r_{c}$ to
$r_{c}+0.01$ with an increment of 0.001, the clustering coefficient increases
from 0.04 to 0.14, and the assortativity one decreases from $-0.3$ to $-0.6$
(see Fig.~\ref{fig:baCharacteristic-raw}). At $r\geq r_{c}$, the dependences
of the coefficients $C$ and $A$ on $r$ appear. The both turn out power-law;
namely, $C\sim{(r-r_{c})}^{\alpha}$, where $\alpha=0.46\pm0.04$, and
$A\sim{(r-r_{c})}^{\gamma}$, where $\gamma=0.26\pm0.04$ (see
Fig.~\ref{fig:rankDistribution-log}).

\subsection{Adjacency matrix for the network with rejection}

Consider the adjacency matrix $A_{ij}$ for a network with rejection. For
convenience, the nodes in the adjacency matrix are enumerated in the order
when the number of node edges decreases. This means that $k_{i}=\sum
\nolimits_{j}A_{ij}$ decreases as $i$ grows.

The change in the network structure at $r\geq r_{c}$ affects the form of the
adjacency matrix as well. For a network with $N=5000$, two adjacency matrices
were constructed: for $r<r_{c}$ (Fig.~\ref{fig:baRankedMatrix-0}) and
$r>r_{c}$ (Fig.~\ref{fig:baRankedMatrix-06}). The both matrices were ranked,
i.e. the network nodes were enumerated in the decreasing order of the edge
number $k_{i}$. From Fig.~\ref{fig:baRankedMatrix-06}, one can see that, in
the case $r>r_{c}$, a considerable square region filled with zeros, which
correspond to unconnected node pairs, appears in the bottom right corner of
the adjacency matrix. This region depends proportionally on the $r$-value.

Hence, such network parameters as the clustering and assortativity
coefficients behave similarly to the order parameter~$\eta$.

\section{Hierarchical $(u,v)$-flower networks with rejection}

Besides random scale-free networks constructed following the
Barab\'{a}si-Albert algorithm and its generalizations, another class of simple
deterministic networks, which are also scale-free, is known \cite{Rozenfeld2}.
These are the so-called $(u,v)$-flowers (Fig. \ref{fig:flowerGraph}).

In this Section, similarly to what was done for the Barab\'{a}si-Albert model,
we generalize the $(u,v)$-flower model by introducing the rejection parameter
$r$ and the stochasticity into the rule of network growth. In this case, the
behavior of network parameters also turns out similar to that of the order
parameter at phase transitions of the second kind.

\subsection{Algorithm for generating $(u,v)$-flowers}

Growing deterministic scale-free networks, which are called $(u,v)$-flower and
$(u,v)$-trees, were proposed and studied in works
\cite{Dor1,Rozenfeld1,Rozenfeld2}. Generally speaking, the enumeration of
nodes in the $(u,v)$-flower can be arbitrary. However, in the example
concerned, the nodes can be enumerated in such a way
(Fig.~\ref{fig:flowerGraph}) that the corresponding adjacency matrix $A_{ij}$
acquires the simplest view. A simple view means in this case such a structure
of $A_{ij}$ matrix that the largest number of the largest square $N\times N$
regions in it remain empty.

Figure~\ref{fig:flowerMatrix} illustrates an $N\times N$ adjacency matrix,
where black cells correspond to matrix members $A_{ij}=1$. At a first step,
the adjacency matrix $\hat{A}$ consists of $3\times3$ elements (from $A_{11}$
to $A_{33}$, the step $t=0$ in Fig.~\ref{fig:flowerGraph}). At a second step,
new elements are added to it, and the matrix consists of $6\times6$ elements
(the step $t=1$ in Fig.~\ref{fig:flowerGraph}). At a third step, new elements
are added, and the matrix consists of $15\times15$ elements (the step $t=2$ in
Fig.~\ref{fig:flowerGraph}). One can see from Fig.~\ref{fig:flowerMatrix} that
the bottom right square of the first step is free of edges and consists of one
element. A bottom right square consisting of $3\times3$ elements is added at
the second step, and a $9\times9$ square at the third one. In
Fig.~\ref{fig:flowerMatrix}, the $A_{ij}=0$ members of adjacency matrix are white-colored.

While comparing our procedure of adjacency matrix generation with that used in
work \cite{Dor1}, the node enumeration order selected in this work allowed us
to obtain additional regions $K_{1}$ and $K_{2}$, in which $A_{ij}=0$ as well.

At every step $t$, there are $N_{t}$ nodes and $L_{t}$ edges \cite{Rozenfeld1}
described by the formulas%
\begin{equation}
N_{t}=(u+v)\cdot N_{t-1}-(u+v),\quad L_{t}=(u+v)^{t}%
.\label{eq:flowerEdgesNodesRecurent}%
\end{equation}
At every step $t$, there appear $N_{t}-N_{t-1}$ nodes and $L_{t}-L_{t-1}$
edges. For instance (see Fig.~\ref{fig:flowerGraph}), 3 nodes and 6 edges
appear at the step $t=1$. According to work \cite{Rozenfeld1}, let us
introduce the notation $w=u+v$ and expand recurrent formulas
(\ref{eq:flowerEdgesNodesRecurent}). We obtain%
\begin{equation}
N_{t}=\frac{w-2}{w-1}\cdot w^{t}+\frac{w}{w-1},\quad L_{t}=w^{t}%
.\label{eq:flowerEdgesNodesOpenRecurent}%
\end{equation}
At every step $t$, there are $\Omega_{t}$ cells which can be filled (see
Fig.~\ref{fig:flowerMatrix}). Their number equals%
\begin{equation}%
\begin{split}
\Omega_{t} &  =(N_{t}-N_{t-1})\cdot N_{t}-N_{t-2}^{2}=\frac{w^{3}-w^{2}%
-1}{w^{4}}N_{t}^{2}+\frac{w^{3}-2w^{2}-2w-2}{w^{3}}N_{t}-\frac{2w^{2}%
+2w+1}{w^{2}}=\\
&  =\frac{w^{2t+4}-5w^{2t+3}+8w^{2t+2}-5w^{2t+1}+4w^{2t}+w^{t+5}%
-3w^{t+4}+4w^{t+2}-w^{5}}{w^{3}\cdot(w-1)^{2}}.
\end{split}
\label{eq:flowerEmpty}%
\end{equation}
Hence, the probability for a matrix cell to be filled equals%
\begin{equation}
W_{t}=\frac{L_{t}-L_{t-1}}{\Omega_{t}}=\frac{w^{t+5}-3w^{t+4}+3w^{t+3}%
-w^{t+2}}{w^{2t+4}-5w^{2t+3}+8w^{2t+2}-5w^{2t+1}+4w^{2t}+w^{t+5}%
-3w^{t+4}+4w^{t+2}-w^{5}}.
\end{equation}
This quantity becomes smaller as the step number grows, so that the adjacency
matrix gets more and more sparse. In work \cite{Dor1}, it was shown that
$(u,v)$-flowers are scale-free networks. For the (1,2)-flower exhibited in
Fig.~\ref{fig:flowerGraph}, the node degree distribution function looks like
$P(k)\sim k^{-(1+\ln{3}/\ln{2})}$. In the general case, for $(u,v)$-flowers we
obtain \cite{Rozenfeld1}%

\begin{equation}
P(k)\sim k^{-\alpha},\quad\alpha=1+\frac{\ln{(u+v)}}{\ln{2}}.
\end{equation}


\subsection{Modification of the $(u,v)$-flower generation algorithm}

Consider the case when not all edges are realized when an $(u,v)$-flower is
generated. Let us also suppose that the lesser are the degrees of nodes, the
higher is the probability that the edge connecting them is absent.

In the case of Barab\'{a}si-Albert model, the rule of attachment with
rejection resulted in the appearance of empty region (the region free of
edges) in the bottom right corner of the adjacency matrix (see
Fig.~\ref{fig:baRankedMatrix-06}). In the case of $(u,v)$-flower model, such a
region already exists. Therefore, empty regions $E_{1}$, $E_{2}$, etc. will be
created in the bottom right corners of already filled regions $\Omega_{1}$,
$\Omega_{2}$, and so on.

The network is generated by filling the regions $\Omega_{t}-E_{t}$ with the
edge number $L_{t}$ in the adjacency matrix. For the random creation of edges
to correspond to the $(u,v)$-flower, the law of node degree distribution
$p\sim k^{-(1+\ln(u+v)/\ln{2})}$ \cite{Rozenfeld2} has to be preserved. For
this purpose, a partitioning of $\Omega_{t}-E_{t}$ regions into 4 equal parts
is used. The probability for the $t$-th part ($t=1\div4$) to be filled equals%
\[
\psi_{t}=\left(  \frac{1}{t}\right)  ^{-(1+\ln(u+v)/\ln{2})}-\psi_{t-1}.
\]
Below, we deal with $(1,2)$-flowers, and the filling probabilities for the
$\Omega_{t}-E_{t}$ regions are 0.37, 0.3, 0.22, and 0.11. Hence, the rejection
manifests itself in the adjacency matrix as $E_{t}$ regions, i.e. empty bottom
right corners of $\Omega$ regions (Fig.~\ref{fig:flowerMatrixExceptive}).

At simulation, empty $E_{t}$ regions proportional to $\Omega_{t}$ ones were
used:%
\begin{equation}
E_{t}=[r\cdot N_{t-1}]\cdot\lbrack r\cdot(N_{t}-N_{t-1})],
\end{equation}
where $r\,N_{t-1}$ is the number of rows, and $r\,(N_{t}-N_{t-1})$ the number
of columns in the $E_{t}$ region (Fig.~\ref{fig:flowerMatrixExceptive}). The
filling probability for the matrix cell equals%
\begin{equation}
W_{t}=\psi_{t}\cdot\frac{L_{t}-L_{t-1}}{\Omega_{t}-E_{t}},
\end{equation}
where $\psi_{t}$ is the filling probability for the $\Omega_{t}-E_{t}$ region.

\subsection{Node degree distribution function}

If the rejection parameter $r=0$, the proposed model transforms into the
standard $(u,v)$-flower one. Similarly to the Barab\'{a}si-Albert model, the
$(u,v)$-flower also contains the threshold value of rejection parameter,
$r_{c}$. If the rejection parameter is less than the threshold value,
$r<r_{c}$, the node degree distribution function $P(k)$ remains to be
power-law, and the network itself remains to be scale-free. But at rejection
parameter values larger than the threshold, $r\geq r_{c}$, the network changes
its structure.

Let us determine the threshold value of rejection parameter. For this purpose,
we calculated the threshold value for all generations of $(1,2)$-flower from
the first to the 14-th one. Analogously to the Barab\'{a}si-Albert model with
rejection, the threshold value also saturated. Now, it occurred at the 8-th
generation, and we obtained $r_{c}=0.75\pm0.04$.

In order to determine the gap magnitude (the parameter $\eta$), we have to
find the bend points (see Fig.~\ref{fig:flowerRank-08}). The sough points
correspond to the minima of the radius of curvature $k=\left(  1+y^{\prime}%
{}^{2}\right)  ^{3/2}/|y^{\prime\prime}|$ \cite{Hazewinkel}, where
$y(x)=a+bx+\dot{c}\arctan(x)+d\,\arctan(\alpha x+\beta)$ is the function
approximating the dependence of the node degree $y$ on the ordinal node number
$x$ \cite{Mills} (see Fig.~\ref{fig:flowerRank-08}).

The parameter $\eta$ behaves similarly to the order parameter in the theory of
phase transitions of the second kind \cite{Landau}. When approaching the
critical rejection value, it decreases following the power law, $\eta
\sim(r-r_{c})^{\beta}$, where $\beta$ is the critical index. $(1,2)$-flower
networks of the eighth generation containing $N=3282$ nodes were simulated. In
Fig.~\ref{fig:flowerRank-gap}, the calculated dependence $\eta=A\,{(r-r_{c}%
)}^{\beta}$, where $\beta=0.28\pm0.05$, is shown.

\subsection{Clustering and assortativity coefficients}

Again, let us consider the behavior of the clustering, $C$, and assortativity,
$A$, coefficients as functions of the rejection parameter $r$.\ At $r\geq
r_{c}$, for the eighth generation of the $(1,2)$-flower network with $N=3282$
nodes, the both do not depend on $r$ and equal $C_{0}\approx0.02$ and
$A_{0}\approx-0.18$, which expectedly coincides with the results of
calculations in works \cite{Rozenfeld1,Rozenfeld2}. At $r\geq r_{c}$, this
dependence appears (as $r$ grows, the clustering coefficient increases and the
assortativity one decreases) and turns out to be power-law; namely,
$C\sim{(r-r_{c})}^{\alpha}$, where $\alpha=0.11\pm0.04$, and $A\sim{(r-r_{c}%
)}^{\gamma}$, where $\gamma=0.08\pm0.04$ (Fig.~\ref{fig:flowerParam-log}).

\section{Conclusions}

A modified rule of preferential attachment, namely, the attachment with
rejection, was proposed for the classes of scale-free networks, the
Barab\'{a}si-Albert and $(u,v)$-flower models. The modification of the
preferential attachment rule consists in the introduction of the rejection
parameter, which rejects some nodes in the course of network growth, i.e.
disables their ability to form new edges with new nodes at the present moment.

The results of numerical simulation showed that the networks belonging to the
considered classes undergo substantial structural changes. The introduction of
a new network parameter, the gap magnitude, and the calculation of already
known characteristics, such as the clustering and assortativity coefficients
and the average least distance between the nodes, allowed a conclusion to be
drawn that the simulated classes of networks undergo the phase transition of
the second kind. The corresponding threshold values of the rejection
parameter, at which the phase transition takes place, were calculated. The
introduced parameter \textquotedblleft gap magnitude\textquotedblright\ was
found to be proportional to the order parameter. The appearance of the gap
$\eta$ in the ranked distribution of network nodes at $r>r_{c}$
(Fig.~\ref{fig:rankDistribution}) may be of interest for economic models
analyzing the redistribution of wealth \cite{Economics2}.

Let us consider the following model describing the distribution of income. Let
every node represent an enterprise. Suppose that the amount of wealth for the
enterprise is proportional to the number of its links with other enterprises,
i.e. to the node degree. Every new node (an enterprise) makes an attachment
(forms a contact) with other, already existing nodes. If the probability of
this contact is proportional to the wealth magnitude (the node degree) of the
enterprise with which the contact is established, the distribution of
enterprises over the magnitudes of their wealth turns out the Pareto
distribution \cite{Economics2, Economics1}, which is observed in a good many
real cases \cite{Economics1}.

However, if the rejection parameter is introduced, a distribution with a
discontinuity (Fig.~\ref{fig:rankDistribution}) rather than the Pareto one is
observed. In terms of the considered economic model (the node degree vs. the
wealth of the enterprise and/or its people), this means that the so-called
middle class disappears. In Fig.~\ref{fig:rankDistribution}, one can see that
there are almost no enterprises with the wealth in the range of $\eta$ (these
are nodes with the degrees approximately from 4 to 70). In other words, there
are only very rich enterprises/people (nodes with large degrees) and poor ones
(nodes with small degrees).

The authors express their gratitude to I.V.Bezsudnov and D.V.~Lande for
numerous useful discussions.

\newpage

\begin{thebibliography}{99}                                                                                               %


\bibitem {Dor2}1

\bibitem {Newman1}1

\bibitem {AlBa2}1

\bibitem {Newman3}1

\bibitem {AlBa3}1

\bibitem {Pareto}1

\bibitem {AlBa1}1

\bibitem {Dor1}1

\bibitem {Rozenfeld2}1

\bibitem {Rozenfeld1}1

\bibitem {Hazewinkel}1

\bibitem {Mills}1

\bibitem {Landau}1

\bibitem {Newman2}1

\bibitem {Economics2}1

\bibitem {Economics1}1
\end{thebibliography}

\newpage

Fig.~1. (a)~Ranked distribution of nodes in the log-log scale for a network
with $N=1000$ nodes at $r=0.6$. The node ordinal numbers are reckoned along
the abscissa axis, and the node degrees along the ordinate one. (b)~The gap
magnitude $\eta$ as a function of the normalized rejection parameter
$(r-r_{c})/r_{c}$ in the interval $r_{c}<r<r_{c}+0.01$ in the log-log scale.

Fig.~2. (a)~Dependences of the clustering, $C$, and assortativity, $A$,
coefficients on the rejection parameter $r$ within the interval $0.4<r<0.7$.
(b)~Dependences of $C$ and $A$ on the normalized rejection parameter
$(r-r_{c})/r_{c}$ in the interval $r_{c}<r<r_{c}+0.06$ in the log-log scale.

Fig.~3.~Adjacency matrices for a network with $N=5000$ nodes at $r=0$ (a) and
0.6 (b). Black cells mark elements $A_{ij}\neq0$.

Fig.~4.~Illustration of the (1,2)-flower generation algorithm at the steps
$t=0$, 1, and 2. Nodes appearing at the current step are marked bold (red online)

Fig.~5.~Adjacency matrices for the 3-rd step of (1,2)-flower generation:
(a)~adjacency matrix of the network with the selected node enumeration;
(b)~its schematic diagram, where $K_{1}$, $K_{2}$, and so on are empty regions

Fig.~6. Schematic diagram of adjacency matrix for the 3-rd step of
(1,2)-flower generation.

Fig.~7. (a)~Ranked distribution of network nodes in the log-log scale at
$r=0.8$. The node ordinal numbers are reckoned along the abscissa axis, and
the node degrees along the ordinate one. (b)~The gap magnitude $\eta$ as a
function of the normalized rejection parameter $(r-r_{c})/r_{c}$ in the
interval $r_{c}<r<r_{c}+0.08$ in the log-log scale.

Fig.~8.~Dependences of the clustering, $C$, and assortativity, $A$,
coefficients on the normalized rejection parameter $(r-r_{c})/r_{c}$ in the
interval $r_{c}<r<r_{c}+0.01$ in the log-log scale.

\newpage

Fig.~\ref{fig:rankDistribution}.

Fig.~\ref{fig:rankDistribution-rank}.

Fig.~\ref{fig:rankDistribution-gap}.

Fig.~\ref{fig:baCharacteristic-raw}

Fig.~\ref{fig:rankDistribution-log}

Fig.~\ref{fig:baRankedMatrix-0}

Fig.~\ref{fig:baRankedMatrix-06}

Fig.~\ref{fig:flowerGraph}

Fig.~\ref{fig:flowerMatrix}

Fig.~\ref{fig:flowerMatrixExceptive}

Fig.~\ref{fig:flowerRank-08}

Fig.~\ref{fig:flowerRank-gap}

Fig.~\ref{fig:flowerParam-log}

\newpage%
%TCIMACRO{\FRAME{ftbpFO}{0.0277in}{0.0277in}{0pt}{\Qct{#1}}{\Qlt{#1}}%
%{Figure}{}}%
%BeginExpansion
%EndExpansion


\dumfig{fig:rankDistribution} \dumfig{fig:rankDistribution-rank}
\dumfig{fig:rankDistribution-gap} \dumfig{fig:baCharacteristic-raw}
\dumfig{fig:rankDistribution-log} \dumfig{fig:baRankedMatrix-0}
\dumfig{fig:baRankedMatrix-06} \dumfig{fig:flowerGraph}
\dumfig{fig:flowerMatrix} \dumfig{fig:flowerMatrixExceptive}
\dumfig{fig:flowerRank-08} \dumfig{fig:flowerRank-gap} \dumfig{fig:flowerParam-log}


\end{document} 